% Options for packages loaded elsewhere
\PassOptionsToPackage{unicode}{hyperref}
\PassOptionsToPackage{hyphens}{url}
%
\documentclass[
]{article}
\usepackage{amsmath,amssymb}
\usepackage{lmodern}
\usepackage{iftex}
\ifPDFTeX
  \usepackage[T1]{fontenc}
  \usepackage[utf8]{inputenc}
  \usepackage{textcomp} % provide euro and other symbols
\else % if luatex or xetex
  \usepackage{unicode-math}
  \defaultfontfeatures{Scale=MatchLowercase}
  \defaultfontfeatures[\rmfamily]{Ligatures=TeX,Scale=1}
\fi
% Use upquote if available, for straight quotes in verbatim environments
\IfFileExists{upquote.sty}{\usepackage{upquote}}{}
\IfFileExists{microtype.sty}{% use microtype if available
  \usepackage[]{microtype}
  \UseMicrotypeSet[protrusion]{basicmath} % disable protrusion for tt fonts
}{}
\makeatletter
\@ifundefined{KOMAClassName}{% if non-KOMA class
  \IfFileExists{parskip.sty}{%
    \usepackage{parskip}
  }{% else
    \setlength{\parindent}{0pt}
    \setlength{\parskip}{6pt plus 2pt minus 1pt}}
}{% if KOMA class
  \KOMAoptions{parskip=half}}
\makeatother
\usepackage{xcolor}
\IfFileExists{xurl.sty}{\usepackage{xurl}}{} % add URL line breaks if available
\IfFileExists{bookmark.sty}{\usepackage{bookmark}}{\usepackage{hyperref}}
\hypersetup{
  hidelinks,
  pdfcreator={LaTeX via pandoc}}
\urlstyle{same} % disable monospaced font for URLs
\usepackage{longtable,booktabs,array}
\usepackage{calc} % for calculating minipage widths
% Correct order of tables after \paragraph or \subparagraph
\usepackage{etoolbox}
\makeatletter
\patchcmd\longtable{\par}{\if@noskipsec\mbox{}\fi\par}{}{}
\makeatother
% Allow footnotes in longtable head/foot
\IfFileExists{footnotehyper.sty}{\usepackage{footnotehyper}}{\usepackage{footnote}}
\makesavenoteenv{longtable}
\usepackage{graphicx}
\makeatletter
\def\maxwidth{\ifdim\Gin@nat@width>\linewidth\linewidth\else\Gin@nat@width\fi}
\def\maxheight{\ifdim\Gin@nat@height>\textheight\textheight\else\Gin@nat@height\fi}
\makeatother
% Scale images if necessary, so that they will not overflow the page
% margins by default, and it is still possible to overwrite the defaults
% using explicit options in \includegraphics[width, height, ...]{}
\setkeys{Gin}{width=\maxwidth,height=\maxheight,keepaspectratio}
% Set default figure placement to htbp
\makeatletter
\def\fps@figure{htbp}
\makeatother
\setlength{\emergencystretch}{3em} % prevent overfull lines
\providecommand{\tightlist}{%
  \setlength{\itemsep}{0pt}\setlength{\parskip}{0pt}}
\setcounter{secnumdepth}{-\maxdimen} % remove section numbering
\ifLuaTeX
  \usepackage{selnolig}  % disable illegal ligatures
\fi

\author{PipaQinse}
\date{\today}

\begin{document}

\hypertarget{ux9009ux62e9ux9898-1}{%
\subsection{选择题}\label{ux9009ux62e9ux9898-1}}

\hypertarget{ux82e5ux96c6ux5408uxamxx4uxauxanx3xux2a7e1uxaux5219uxamux2229n}{%
\paragraph{\texorpdfstring{若集合\(M=\{x|\sqrt x<4\}\),\(N=\{x|3x \geqslant 1\}\),则\(M \cap N=\)}{若集合M=\textbackslash\{x\textbar\textbackslash sqrt x\textless4\textbackslash\},N=\textbackslash\{x\textbar3x \textbackslash geqslant 1\textbackslash\},则M \textbackslash cap N=}}\label{ux82e5ux96c6ux5408uxamxx4uxauxanx3xux2a7e1uxaux5219uxamux2229n}}

\begin{longtable}[]{@{}ll@{}}
\toprule
& \\
\midrule
\endhead
A. & \(\{x|0<x<2\}\) \\
B. & \(\{x|{1 \over 3}<x<2\}\) \\
C. & \(\{x\|3<x<16\}\) \\
D. & \(\{x|{1 \over 3}<x<16\}\) \\
\bottomrule
\end{longtable}

\hypertarget{ux82e5uxai1ux2212z1uxaux5219uxazzuxaf}{%
\paragraph{\texorpdfstring{若\({\rm i}(1-z)=1\),则\(z+\bar z=\)}{若\{\textbackslash rm i\}(1-z)=1,则z+\textbackslash bar z=}}\label{ux82e5uxai1ux2212z1uxaux5219uxazzuxaf}}

\begin{longtable}[]{@{}ll@{}}
\toprule
& \\
\midrule
\endhead
A. & \(-2\) \\
B. & \(-1 \) \\
C. & \(1\) \\
D. & \(2\) \\
\bottomrule
\end{longtable}

\hypertarget{ux5728uxaux25b3abcuxaux4e2dux70b9uxaduxaux5728ux8fb9uxaabuxaux4e0auxabd2dauxaux8bb0uxacamuxauxacdnuxaux5219uxacb}{%
\paragraph{\texorpdfstring{在\(\triangle ABC\)中,点\(D\)在边\(AB\)上,\(BD=2DA\),记\(CA=\boldsymbol m\),\(CD=\boldsymbol n\),则\(CB=\)}{在\textbackslash triangle ABC中,点D在边AB上,BD=2DA,记CA=\textbackslash boldsymbol m,CD=\textbackslash boldsymbol n,则CB=}}\label{ux5728uxaux25b3abcuxaux4e2dux70b9uxaduxaux5728ux8fb9uxaabuxaux4e0auxabd2dauxaux8bb0uxacamuxauxacdnuxaux5219uxacb}}

\begin{longtable}[]{@{}ll@{}}
\toprule
& \\
\midrule
\endhead
A. & \(3\boldsymbol m-2\boldsymbol n\) \\
B. & \(-2\boldsymbol m+3\boldsymbol n\) \\
C. & \(3\boldsymbol m+2\boldsymbol n\) \\
D. & \(2\boldsymbol m+3\boldsymbol n\) \\
\bottomrule
\end{longtable}

\hypertarget{ux5357ux6c34ux5317ux8c03ux5de5ux7a0bux7f13ux89e3ux4e86ux5317ux65b9ux4e00ux4e9bux5730ux533aux6c34ux8d44ux6e90ux77edux7f3aux95eeux9898ux5176ux4e2dux4e00ux90e8ux5206ux6c34ux84c4ux5165ux67d0ux6c34ux5e93ux5df2ux77e5ux8be5ux6c34ux5e93ux6c34ux4f4dux4e3aux6d77ux62d4uxa1485muxaux65f6ux76f8ux5e94ux6c34ux9762ux7684ux9762ux79efux4e3auxa1400km2uxaux6c34ux4f4dux4e3aux6d77ux62d4uxa1575muxaux65f6ux76f8ux5e94ux6c34ux9762ux7684ux9762ux79efux4e3auxa1800km2uxaux5c06ux8be5ux6c34ux5e93ux5728ux8fd9ux4e24ux4e2aux6c34ux4f4dux95f4ux7684ux5f62ux72b6ux770bux4f5cux4e00ux4e2aux68f1ux53f0ux5219ux8be5ux6c34ux5e93ux6c34ux4f4dux4eceux6d77ux62d4uxa1485muxaux4e0aux5347ux5230uxa1575muxaux65f6ux589eux52a0ux7684ux6c34ux91cfux7ea6ux4e3a}{%
\paragraph{\texorpdfstring{南水北调工程缓解了北方一些地区水资源短缺问题,其中一部分水蓄入某水库,已知该水库水位为海拔\(148.5 \rm m\)时,相应水面的面积为\(140.0\rm km^2\);水位为海拔\(157.5\rm m\)时,相应水面的面积为\(180.0\rm km^2\)。将该水库在这两个水位间的形状看作一个棱台则该水库水位从海拔\(148.5 \rm m\)上升到\(157.5\rm m\)时,增加的水量约为}{南水北调工程缓解了北方一些地区水资源短缺问题,其中一部分水蓄入某水库,已知该水库水位为海拔148.5 \textbackslash rm m时,相应水面的面积为140.0\textbackslash rm km\^{}2;水位为海拔157.5\textbackslash rm m时,相应水面的面积为180.0\textbackslash rm km\^{}2。将该水库在这两个水位间的形状看作一个棱台则该水库水位从海拔148.5 \textbackslash rm m上升到157.5\textbackslash rm m时,增加的水量约为}}\label{ux5357ux6c34ux5317ux8c03ux5de5ux7a0bux7f13ux89e3ux4e86ux5317ux65b9ux4e00ux4e9bux5730ux533aux6c34ux8d44ux6e90ux77edux7f3aux95eeux9898ux5176ux4e2dux4e00ux90e8ux5206ux6c34ux84c4ux5165ux67d0ux6c34ux5e93ux5df2ux77e5ux8be5ux6c34ux5e93ux6c34ux4f4dux4e3aux6d77ux62d4uxa1485muxaux65f6ux76f8ux5e94ux6c34ux9762ux7684ux9762ux79efux4e3auxa1400km2uxaux6c34ux4f4dux4e3aux6d77ux62d4uxa1575muxaux65f6ux76f8ux5e94ux6c34ux9762ux7684ux9762ux79efux4e3auxa1800km2uxaux5c06ux8be5ux6c34ux5e93ux5728ux8fd9ux4e24ux4e2aux6c34ux4f4dux95f4ux7684ux5f62ux72b6ux770bux4f5cux4e00ux4e2aux68f1ux53f0ux5219ux8be5ux6c34ux5e93ux6c34ux4f4dux4eceux6d77ux62d4uxa1485muxaux4e0aux5347ux5230uxa1575muxaux65f6ux589eux52a0ux7684ux6c34ux91cfux7ea6ux4e3a}}

\begin{longtable}[]{@{}ll@{}}
\toprule
& \\
\midrule
\endhead
A. & \(1.0\times10^9\rm m^3\) \\
B. & \(1,2\times10^9\rm m^3\) \\
C. & \(1.4\times10^9\rm m^3\) \\
D. & \(1.6\times10^9\rm m^3\) \\
\bottomrule
\end{longtable}

\hypertarget{ux4eceuxa2uxaux81f3uxa8uxaux7684uxa7uxaux4e2aux6574ux6570ux4e2dux968fux673aux53d6uxa2uxaux4e2aux4e0dux540cux7684ux6570ux5219ux8fd9uxa2uxaux4e2aux6570ux4e92ux8d28ux7684ux6982ux7387ux4e3a}{%
\paragraph{\texorpdfstring{从\(2\)至\(8\)的\(7\)个整数中随机取\(2\)个不同的数,则这\(2\)个数互质的概率为}{从2至8的7个整数中随机取2个不同的数,则这2个数互质的概率为}}\label{ux4eceuxa2uxaux81f3uxa8uxaux7684uxa7uxaux4e2aux6574ux6570ux4e2dux968fux673aux53d6uxa2uxaux4e2aux4e0dux540cux7684ux6570ux5219ux8fd9uxa2uxaux4e2aux6570ux4e92ux8d28ux7684ux6982ux7387ux4e3a}}

\begin{longtable}[]{@{}ll@{}}
\toprule
& \\
\midrule
\endhead
A. & \({1 \over 6}\) \\
B. & \({1 \over 3}\) \\
C. & \({1 \over 2}\) \\
D. & \({2 \over 3}\) \\
\bottomrule
\end{longtable}

\hypertarget{ux8bb0ux51fdux6570uxafxsinux3c9xux3c04bux3c90uxaux7684ux6700ux5c0fux6b63ux5468ux671fux4e3auxatuxaux82e5uxa23ux3c0tux3c0uxaux4e14uxayfxuxaux7684ux56feux50cfux5173ux4e8eux70b9uxa32ux3c02uxaux4e2dux5fc3ux5bf9ux79f0ux5219uxafux3c02}{%
\paragraph{\texorpdfstring{记函数\(f(x)=\sin(\omega x + {\pi \over 4})+b (\omega > 0)\)的最小正周期为\(T\),若\({2 \over 3}\pi < T <\pi\),且\(y=f(x)\)的图像关于点\(({3 \over 2}\pi,2)\)中心对称,则\(f({\pi \over 2})=\)}{记函数f(x)=\textbackslash sin(\textbackslash omega x + \{\textbackslash pi \textbackslash over 4\})+b (\textbackslash omega \textgreater{} 0)的最小正周期为T,若\{2 \textbackslash over 3\}\textbackslash pi \textless{} T \textless\textbackslash pi,且y=f(x)的图像关于点(\{3 \textbackslash over 2\}\textbackslash pi,2)中心对称,则f(\{\textbackslash pi \textbackslash over 2\})=}}\label{ux8bb0ux51fdux6570uxafxsinux3c9xux3c04bux3c90uxaux7684ux6700ux5c0fux6b63ux5468ux671fux4e3auxatuxaux82e5uxa23ux3c0tux3c0uxaux4e14uxayfxuxaux7684ux56feux50cfux5173ux4e8eux70b9uxa32ux3c02uxaux4e2dux5fc3ux5bf9ux79f0ux5219uxafux3c02}}

\begin{longtable}[]{@{}ll@{}}
\toprule
& \\
\midrule
\endhead
A. & \(1\) \\
B. & \(3 \over 2\) \\
C. & \(5 \over 2\) \\
D. & \(3\) \\
\bottomrule
\end{longtable}

\hypertarget{ux8bbeuxaa01e01uxauxab19uxauxacln09uxaux5219}{%
\paragraph{\texorpdfstring{设\(a=0.1e^{0.1}\),\(b={1 \over 9}\),\(c=\ln0.9\),则}{设a=0.1e\^{}\{0.1\},b=\{1 \textbackslash over 9\},c=\textbackslash ln0.9,则}}\label{ux8bbeuxaa01e01uxauxab19uxauxacln09uxaux5219}}

\begin{longtable}[]{@{}ll@{}}
\toprule
& \\
\midrule
\endhead
A. & \(a<b<c\) \\
B. & \(c<b<a\) \\
C. & \(c<a<b\) \\
D. & \(a<c<b\) \\
\bottomrule
\end{longtable}

\hypertarget{ux5df2ux77e5ux6b63ux56dbux68f1ux9525ux7684ux4fa7ux68f1ux957fux4e3auxa1uxaux5176ux5404ux9876ux70b9ux90fdux5728ux540cux4e00ux7403ux9762ux4e0aux82e5ux8be5ux7403ux7684ux4f53ux79efux4e3auxa36ux3c0uxaux4e14uxa3l33uxaux5219ux8be5ux6b63ux56dbux697cux9525ux4f53ux79efux7684ux53d6ux503cux8303ux56f4ux662f}{%
\paragraph{\texorpdfstring{已知正四棱锥的侧棱长为\(1\),其各顶点都在同一球面上,若该球的体积为\(36\pi\),且\(3<l<3\sqrt3\),则该正四楼锥体积的取值范围是}{已知正四棱锥的侧棱长为1,其各顶点都在同一球面上,若该球的体积为36\textbackslash pi,且3\textless l\textless3\textbackslash sqrt3,则该正四楼锥体积的取值范围是}}\label{ux5df2ux77e5ux6b63ux56dbux68f1ux9525ux7684ux4fa7ux68f1ux957fux4e3auxa1uxaux5176ux5404ux9876ux70b9ux90fdux5728ux540cux4e00ux7403ux9762ux4e0aux82e5ux8be5ux7403ux7684ux4f53ux79efux4e3auxa36ux3c0uxaux4e14uxa3l33uxaux5219ux8be5ux6b63ux56dbux697cux9525ux4f53ux79efux7684ux53d6ux503cux8303ux56f4ux662f}}

\begin{longtable}[]{@{}ll@{}}
\toprule
& \\
\midrule
\endhead
A. & \([18,{81 \over 4}]\) \\
B. & \([{27 \over 4},{81 \over 4}]\) \\
C. & \([{27 \over 4},{64 \over 3}]\) \\
D. & \([18,27]\) \\
\bottomrule
\end{longtable}

\hypertarget{ux9009ux62e9ux9898-2}{%
\subsection{选择题}\label{ux9009ux62e9ux9898-2}}

\hypertarget{ux5df2ux77e5ux6b63ux65b9ux4f53uxaabcdux2212a1b1c1d1uxaux5219}{%
\paragraph{\texorpdfstring{已知正方体\(ABCD-A_1B_1C_1D_1\),则}{已知正方体ABCD-A\_1B\_1C\_1D\_1,则}}\label{ux5df2ux77e5ux6b63ux65b9ux4f53uxaabcdux2212a1b1c1d1uxaux5219}}

\begin{longtable}[]{@{}ll@{}}
\toprule
& \\
\midrule
\endhead
A. & 直线\(BC_1\)与\(DA_1\)所成的角为\(90^\circ\) \\
B. & 直线\(BC_1\)与\(CA_1\)所成的角为\(90^\circ\) \\
C. & 直线\(BC_1\)与平面\(BB_1D_1D\)所成的角为\(45^\circ\) \\
D. & 直线\(BC_1\)与平面\(ABCD\)所成的角为\(45^\circ\) \\
\bottomrule
\end{longtable}

\hypertarget{ux5df2ux77e5ux51fdux6570uxafxx3ux2212x1uxaux5219}{%
\paragraph{\texorpdfstring{已知函数\(f(x)=x^3-x+1\),则}{已知函数f(x)=x\^{}3-x+1,则}}\label{ux5df2ux77e5ux51fdux6570uxafxx3ux2212x1uxaux5219}}

\begin{longtable}[]{@{}ll@{}}
\toprule
& \\
\midrule
\endhead
A. & \(f(x)\)有两个极值点 \\
B. & \(f(x)\)有三个零点 \\
C. & 点\((0,1)\)是曲线\(y=f(x)\)的对称中心 \\
D. & 直线\(y=2x\)是曲线\(y=f(x)\)的切线 \\
\bottomrule
\end{longtable}

\hypertarget{ux5df2ux77e5uxaouxaux4e3aux5750ux6807ux539fux70b9ux70b9uxaa11uxaux5728ux629bux7269ux7ebfuxacx22pyp0uxaux4e0aux8fc7ux70b9uxab0ux22121uxaux7684ux76f4ux7ebfux4ea4uxacuxaux4e8euxapuxauxaquxaux4e24ux70b9ux5219}{%
\paragraph{\texorpdfstring{已知\(O\)为坐标原点,点\(A(1,1)\)在抛物线\(C:x^2=2py (p>0)\)上,过点\(B(0,-1)\)的直线交\(C\)于\(P\),\(Q\)两点,则}{已知O为坐标原点,点A(1,1)在抛物线C:x\^{}2=2py (p\textgreater0)上,过点B(0,-1)的直线交C于P,Q两点,则}}\label{ux5df2ux77e5uxaouxaux4e3aux5750ux6807ux539fux70b9ux70b9uxaa11uxaux5728ux629bux7269ux7ebfuxacx22pyp0uxaux4e0aux8fc7ux70b9uxab0ux22121uxaux7684ux76f4ux7ebfux4ea4uxacuxaux4e8euxapuxauxaquxaux4e24ux70b9ux5219}}

\begin{longtable}[]{@{}ll@{}}
\toprule
& \\
\midrule
\endhead
A. & \(C\)的准线为\(y=-1\) \\
B. & 直线\(AB\)与\(C\)相切 \\
C. & \(\left| OP \right| \left| OQ \right| > \left| OA \right| ^2\) \\
D. & \(\left| BP \right| \left| BQ \right| > \left| BA \right| ^2\) \\
\bottomrule
\end{longtable}

\hypertarget{ux5df2ux77e5ux51fdux6570uxafxuxaux53caux5176ux5bfcux51fdux6570uxafxuxaux7684ux5b9aux4e49ux57dfux5747ux4e3auxaruxaux8bb0uxagxfxuxaux82e5uxaf32ux22122xuxauxag2xuxaux5747ux4e3aux5076ux51fdux6570ux5219}{%
\paragraph{\texorpdfstring{已知函数\(f(x)\)及其导函数\(f^\prime(x)\)的定义域均为\(\Bbb R\),记\(g(x)=f^\prime(x)\),若\(f({3 \over 2}-2x)\),\(g(2+.x)\)均为偶函数,则}{已知函数f(x)及其导函数f\^{}\textbackslash prime(x)的定义域均为\textbackslash Bbb R,记g(x)=f\^{}\textbackslash prime(x),若f(\{3 \textbackslash over 2\}-2x),g(2+.x)均为偶函数,则}}\label{ux5df2ux77e5ux51fdux6570uxafxuxaux53caux5176ux5bfcux51fdux6570uxafxuxaux7684ux5b9aux4e49ux57dfux5747ux4e3auxaruxaux8bb0uxagxfxuxaux82e5uxaf32ux22122xuxauxag2xuxaux5747ux4e3aux5076ux51fdux6570ux5219}}

\begin{longtable}[]{@{}ll@{}}
\toprule
& \\
\midrule
\endhead
A. & \(f(0)=0\) \\
B. & \(g(-{1 \over 2})=0\) \\
C. & \(f(-1)=f(4)\) \\
D. & \(g(-l)=g(2)\) \\
\bottomrule
\end{longtable}

\hypertarget{ux586bux7a7aux9898}{%
\subsection{填空题}\label{ux586bux7a7aux9898}}

\hypertarget{1ux2212xyxy8uxaux7684ux5c55ux5f00ux5f0fux4e2duxax2y6uxaux7684ux7cfbux6570ux7528ux6570ux5b57ux4f5cux7b54}{%
\paragraph{\texorpdfstring{\((1-{x \over y})(x + y)^8\)的展开式中\(x^2y^6\)的系数(用数字作答),}{(1-\{x \textbackslash over y\})(x + y)\^{}8的展开式中x\^{}2y\^{}6的系数(用数字作答),}}\label{1ux2212xyxy8uxaux7684ux5c55ux5f00ux5f0fux4e2duxax2y6uxaux7684ux7cfbux6570ux7528ux6570ux5b57ux4f5cux7b54}}

\hypertarget{ux5199ux51faux4e0eux5706uxax2y21uxaux548cuxaxux221232yux22124216uxaux90fdux76f8ux5207ux7684ux4e00ux6761ux76f4ux7ebfux7684ux65b9ux7a0b}{%
\paragraph{\texorpdfstring{写出与圆\(x^2 + y^2 = 1\)和\((x-3)^2 + (y-4)^2 =16\)都相切的一条直线的方程}{写出与圆x\^{}2 + y\^{}2 = 1和(x-3)\^{}2 + (y-4)\^{}2 =16都相切的一条直线的方程}}\label{ux5199ux51faux4e0eux5706uxax2y21uxaux548cuxaxux221232yux22124216uxaux90fdux76f8ux5207ux7684ux4e00ux6761ux76f4ux7ebfux7684ux65b9ux7a0b}}

\hypertarget{ux82e5ux66f2ux7ebfuxayxaexuxaux6709ux4e24ux6761ux8fc7ux5750ux6807ux539fux70b9ux7684ux5207ux7ebfux5219uxaauxaux7684ux53d6ux503cux8303ux56f4ux662f}{%
\paragraph{\texorpdfstring{若曲线\(y=(x+a){\rm e}^x\)有两条过坐标原点的切线,则\(a\)的取值范围是}{若曲线y=(x+a)\{\textbackslash rm e\}\^{}x有两条过坐标原点的切线,则a的取值范围是}}\label{ux82e5ux66f2ux7ebfuxayxaexuxaux6709ux4e24ux6761ux8fc7ux5750ux6807ux539fux70b9ux7684ux5207ux7ebfux5219uxaauxaux7684ux53d6ux503cux8303ux56f4ux662f}}

\hypertarget{ux5df2ux77e5ux692dux5706uxacx2a2y2b21ab0uxauxacuxaux7684ux4e0aux9876ux70b9ux4e3auxaauxaux4e24ux4e2aux7126ux70b9ux4e3auxaf1uxauxaf2uxaux79bbux5fc3ux7387ux4e3auxa12uxaux8fc7uxaf1uxaux4e14ux5782ux76f4ux4e8euxaaf2uxaux7684ux76f4ux7ebfux4e0euxacuxaux4ea4ux4e8euxaduxauxaeuxaux4e24ux70b9uxade6uxaux5219uxaux25b3adeuxaux7684ux5468ux957fux662f}{%
\paragraph{\texorpdfstring{已知椭圆\(C:{x^2 \over a^2} + {y^2 \over b^2} = 1 (a>b>0)\),\(C\)的上顶点为\(A\),两个焦点为\(F_1\),\(F_2\),离心率为\(1 \over 2\),过\(F_1\)且垂直于\(AF_2\)的直线与\(C\)交于\(D\),\(E\)两点,\(\left| DE \right| = 6\),则\(\triangle ADE\)的周长是}{已知椭圆C:\{x\^{}2 \textbackslash over a\^{}2\} + \{y\^{}2 \textbackslash over b\^{}2\} = 1 (a\textgreater b\textgreater0),C的上顶点为A,两个焦点为F\_1,F\_2,离心率为1 \textbackslash over 2,过F\_1且垂直于AF\_2的直线与C交于D,E两点,\textbackslash left\textbar{} DE \textbackslash right\textbar{} = 6,则\textbackslash triangle ADE的周长是}}\label{ux5df2ux77e5ux692dux5706uxacx2a2y2b21ab0uxauxacuxaux7684ux4e0aux9876ux70b9ux4e3auxaauxaux4e24ux4e2aux7126ux70b9ux4e3auxaf1uxauxaf2uxaux79bbux5fc3ux7387ux4e3auxa12uxaux8fc7uxaf1uxaux4e14ux5782ux76f4ux4e8euxaaf2uxaux7684ux76f4ux7ebfux4e0euxacuxaux4ea4ux4e8euxaduxauxaeuxaux4e24ux70b9uxade6uxaux5219uxaux25b3adeuxaux7684ux5468ux957fux662f}}

\hypertarget{ux89e3ux7b54ux9898}{%
\subsection{解答题}\label{ux89e3ux7b54ux9898}}

\hypertarget{ux8bb0uxasuxaux4e3aux6570ux5217uxaanuxaux7684ux524duxanuxaux9879ux548cux5df2ux77e5uxaa11uxauxasnanuxaux662fux516cux5deeux4e3auxa13uxaux7684ux7b49ux5deeux6559ux5217}{%
\paragraph{\texorpdfstring{记\(S\)为数列\(\{a_n\}\)的前\(n\)项和,已知\(a_1=1\),\(\{{S_n \over a_n}\}\)是公差为\({1 \over 3}\)的等差教列.}{记S为数列\textbackslash\{a\_n\textbackslash\}的前n项和,已知a\_1=1,\textbackslash\{\{S\_n \textbackslash over a\_n\}\textbackslash\}是公差为\{1 \textbackslash over 3\}的等差教列.}}\label{ux8bb0uxasuxaux4e3aux6570ux5217uxaanuxaux7684ux524duxanuxaux9879ux548cux5df2ux77e5uxaa11uxauxasnanuxaux662fux516cux5deeux4e3auxa13uxaux7684ux7b49ux5deeux6559ux5217}}

\hypertarget{ux6c42uxaanuxaux7684ux901aux9879ux516cux5f0f}{%
\subparagraph{\texorpdfstring{求\({a_n}\)的通项公式;}{求\{a\_n\}的通项公式;}}\label{ux6c42uxaanuxaux7684ux901aux9879ux516cux5f0f}}

\hypertarget{ux8bc1ux660euxa1a11a2ux22ef1an2uxa}{%
\subparagraph{\texorpdfstring{证明:\({1 \over a_1} + {1 \over a_2} + \cdots + {1 \over a_n} < 2\).}{证明:\{1 \textbackslash over a\_1\} + \{1 \textbackslash over a\_2\} + \textbackslash cdots + \{1 \textbackslash over a\_n\} \textless{} 2.}}\label{ux8bc1ux660euxa1a11a2ux22ef1an2uxa}}

\hypertarget{ux8bb0uxaux25b3abcuxaux7684ux5185ux89d2uxaauxauxabuxauxacuxaux7684ux5bf9ux8fb9ux5206ux522bux4e3auxaauxauxabuxauxacuxaux5df2ux77e5uxacosa1sinasin2b1cos2buxa}{%
\paragraph{\texorpdfstring{记\(\triangle ABC\)的内角\(A\),\(B\),\(C\)的对边分别为\(a\),\(b\),\(c\),已知\({\cos A \over 1 + \sin A} + {\sin 2B \over 1+\cos 2B}\).}{记\textbackslash triangle ABC的内角A,B,C的对边分别为a,b,c,已知\{\textbackslash cos A \textbackslash over 1 + \textbackslash sin A\} + \{\textbackslash sin 2B \textbackslash over 1+\textbackslash cos 2B\}.}}\label{ux8bb0uxaux25b3abcuxaux7684ux5185ux89d2uxaauxauxabuxauxacuxaux7684ux5bf9ux8fb9ux5206ux522bux4e3auxaauxauxabuxauxacuxaux5df2ux77e5uxacosa1sinasin2b1cos2buxa}}

\hypertarget{ux82e5uxac23ux3c0uxaux6c42uxabuxa}{%
\subparagraph{\texorpdfstring{若\(C = {2 \over 3}\pi\),求\(B\);}{若C = \{2 \textbackslash over 3\}\textbackslash pi,求B;}}\label{ux82e5uxac23ux3c0uxaux6c42uxabuxa}}

\hypertarget{ux6c42uxaa2b2c2uxaux7684ux6700ux5c0fux503c}{%
\subparagraph{\texorpdfstring{求\(a^2 + b^2 \over c^2\)的最小值.}{求a\^{}2 + b\^{}2 \textbackslash over c\^{}2的最小值.}}\label{ux6c42uxaa2b2c2uxaux7684ux6700ux5c0fux503c}}

\hypertarget{ux5982ux56feux76f4ux4e09ux68f1ux67f1uxaabcux2212a1b1c1uxaux7684ux4f53ux79efux4e3a4uxaux25b3abcuxaux7684ux9762ux79efux4e3auxa22}{%
\paragraph{\texorpdfstring{如图,直三棱柱\(ABC-A_1B_1C_1\)的体积为4,\(\triangle ABC\)的面积为\(2\sqrt2\)}{如图,直三棱柱ABC-A\_1B\_1C\_1的体积为4,\textbackslash triangle ABC的面积为2\textbackslash sqrt2}}\label{ux5982ux56feux76f4ux4e09ux68f1ux67f1uxaabcux2212a1b1c1uxaux7684ux4f53ux79efux4e3a4uxaux25b3abcuxaux7684ux9762ux79efux4e3auxa22}}

\hypertarget{ux6c42uxaauxaux5230ux5e73ux9762uxaabcuxaux7684ux8dddux79bb}{%
\subparagraph{\texorpdfstring{求\(A\)到平面\(ABC\)的距离;}{求A到平面ABC的距离;}}\label{ux6c42uxaauxaux5230ux5e73ux9762uxaabcuxaux7684ux8dddux79bb}}

\hypertarget{ux8bbeuxaduxaux4e3auxaa1cuxaux7684ux4e2dux70b9uxaaa1abuxaux5e73ux9762uxaa1b1c1uxaux22a5uxaux5e73ux9762uxaabb1a1uxaux6c42ux4e8cux9762ux89d2uxaaux2212bdux2212cuxaux7684ux6b63ux5f26ux503c}{%
\subparagraph{\texorpdfstring{设\(D\)为\(A_1C\)的中点,\(AA_1 = AB\),平面\(A_1B_1C_1\)\(\perp\)平面\(ABB_1A_1\),求二面角\(A-BD-C\)的正弦值.}{设D为A\_1C的中点,AA\_1 = AB,平面A\_1B\_1C\_1\textbackslash perp平面ABB\_1A\_1,求二面角A-BD-C的正弦值.}}\label{ux8bbeuxaduxaux4e3auxaa1cuxaux7684ux4e2dux70b9uxaaa1abuxaux5e73ux9762uxaa1b1c1uxaux22a5uxaux5e73ux9762uxaabb1a1uxaux6c42ux4e8cux9762ux89d2uxaaux2212bdux2212cuxaux7684ux6b63ux5f26ux503c}}

\begin{figure}
\centering
\includegraphics{https://docimg9.docs.qq.com/image/xKoI0kX60kHj86beD0sAZg.jpeg?w=100\&h=107}
\caption{}
\end{figure}

\hypertarget{ux4e00ux533bux7597ux56e2ux961fux4e3aux7814ux7a76ux67d0ux5730ux7684ux4e00ux79cdux5730ux65b9ux6027ux75beux75c5ux4e0eux5f53ux5730ux5c45ux6c11ux7684ux536bux751fux4e60ux60efux536bux751fux4e60ux60efux5206ux4e3aux826fux597dux548cux4e0dux591fux826fux597dux4e24ux7c7buxff09ux7684ux5173ux7cfbux5728ux5df2ux60a3ux8be5ux75beux75c5ux7684ux75c5ux4f8bux4e2dux968fux673aux8c03ux67e5ux4e86100ux4f8bux79f0ux4e3aux75c5ux4f8bux7ec4uxff09ux540cux65f6ux5728ux672aux60a3ux8be5ux75beux75c5ux7684ux4ebaux7fa4ux4e2dux968fux673aux8c03ux67e5ux4e86100ux4ebaux505aux4e3aux5bf9ux7167ux7ec4uxff09ux5f97ux5230ux5982ux4e0bux6570ux636e}{%
\paragraph{一医疗团队为研究某地的一种地方性疾病与当地居民的卫生习惯(卫生习惯分为良好和不够良好两类)的关系,在已患该疾病的病例中随机调查了100例(称为病例组),同时在未患该疾病的人群中随机调查了100人(做为对照组),得到如下数据:}\label{ux4e00ux533bux7597ux56e2ux961fux4e3aux7814ux7a76ux67d0ux5730ux7684ux4e00ux79cdux5730ux65b9ux6027ux75beux75c5ux4e0eux5f53ux5730ux5c45ux6c11ux7684ux536bux751fux4e60ux60efux536bux751fux4e60ux60efux5206ux4e3aux826fux597dux548cux4e0dux591fux826fux597dux4e24ux7c7buxff09ux7684ux5173ux7cfbux5728ux5df2ux60a3ux8be5ux75beux75c5ux7684ux75c5ux4f8bux4e2dux968fux673aux8c03ux67e5ux4e86100ux4f8bux79f0ux4e3aux75c5ux4f8bux7ec4uxff09ux540cux65f6ux5728ux672aux60a3ux8be5ux75beux75c5ux7684ux4ebaux7fa4ux4e2dux968fux673aux8c03ux67e5ux4e86100ux4ebaux505aux4e3aux5bf9ux7167ux7ec4uxff09ux5f97ux5230ux5982ux4e0bux6570ux636e}}

\begin{longtable}[]{@{}lll@{}}
\toprule
& 不够良好 & 良好 \\
\midrule
\endhead
病例组 & 40 & 60 \\
对照组 & 10 & 90 \\
\bottomrule
\end{longtable}

\hypertarget{ux80fdux5426ux6709uxa99uxaux7684ux628aux636eux8ba4ux4e3aux60a3ux8be5ux75beux75c5ux7fa4ux4f11ux4e0eux672aux611fux8be5ux75beux75c5ux7fa4ux4f53ux7684ux536bux751fux4e60ux60efux6709ux5deeux5f02}{%
\subparagraph{\texorpdfstring{能否有\(99\%\)的把据认为患该疾病群休与未感该疾病群体的卫生习惯有差异?}{能否有99\textbackslash\%的把据认为患该疾病群休与未感该疾病群体的卫生习惯有差异?}}\label{ux80fdux5426ux6709uxa99uxaux7684ux628aux636eux8ba4ux4e3aux60a3ux8be5ux75beux75c5ux7fa4ux4f11ux4e0eux672aux611fux8be5ux75beux75c5ux7fa4ux4f53ux7684ux536bux751fux4e60ux60efux6709ux5deeux5f02}}

\hypertarget{ux4eceux8be5ux5730ux7684ux4ebaux7fa4ux4e2dux4efbux9009ux4e00ux4ebauxaauxaux8868ux793aux4e8bux4ef6ux9009ux5230ux7684ux4ebaux536bux751fux4e60ux60efux4e0dux591fux826fux597duxabuxaux8868ux793aux4e8bux4ef6ux9009ux5230ux7684ux4ebaux60a3ux6709ux8be5ux75beux75c5uxapbapbuxafauxaux4e0euxapbauxafpbuxafauxafuxaux7684ux6bd4ux503cux662fux536bux751fux4e60ux60efux4e0dux591fux826fux597dux5bf9ux60a3ux8be5ux75beux75c5ux98ceux9669ux7a0bux5ea6ux7684ux4e00ux9879ux5ea6ux91cfux6307ux6807ux8bb0ux8be5ux6307ux6807ux4e3auxaruxa}{%
\subparagraph{\texorpdfstring{从该地的人群中任选一人,\(A\)表示事件``选到的人卫生习惯不够良好'',\(B\)表示事件``选到的人患有该疾病'',\(P(B|A) \over P(\bar B|A)\)与\(P(B|\bar A) \over P(\bar B|\bar A)\)的比值是卫生习惯不够良好对患该疾病风险程度的一项度量指标,记该指标为\(R\).}{从该地的人群中任选一人,A表示事件``选到的人卫生习惯不够良好'',B表示事件``选到的人患有该疾病'',P(B\textbar A) \textbackslash over P(\textbackslash bar B\textbar A)与P(B\textbar\textbackslash bar A) \textbackslash over P(\textbackslash bar B\textbar\textbackslash bar A)的比值是卫生习惯不够良好对患该疾病风险程度的一项度量指标,记该指标为R.}}\label{ux4eceux8be5ux5730ux7684ux4ebaux7fa4ux4e2dux4efbux9009ux4e00ux4ebauxaauxaux8868ux793aux4e8bux4ef6ux9009ux5230ux7684ux4ebaux536bux751fux4e60ux60efux4e0dux591fux826fux597duxabuxaux8868ux793aux4e8bux4ef6ux9009ux5230ux7684ux4ebaux60a3ux6709ux8be5ux75beux75c5uxapbapbuxafauxaux4e0euxapbauxafpbuxafauxafuxaux7684ux6bd4ux503cux662fux536bux751fux4e60ux60efux4e0dux591fux826fux597dux5bf9ux60a3ux8be5ux75beux75c5ux98ceux9669ux7a0bux5ea6ux7684ux4e00ux9879ux5ea6ux91cfux6307ux6807ux8bb0ux8be5ux6307ux6807ux4e3auxaruxa}}

证明:

\[R = {P(B|A) \over P(\bar B|A)}{ P(\bar B|\bar A) \over P(B|\bar A)}\]

利用该调查数据,给出\(P(A|B)\),\(P(A|\bar B)\)的估计值,并利用(i)的结果给出\(R\)的估计值.

附:

\[K^2 = { n(ad - bc) \over (a+b)(c+d)(a+c)(b+d)}\]

\begin{longtable}[]{@{}llll@{}}
\toprule
\(P(K^2 \geqslant k)\) & 0.050 & 0.010 & 0.001 \\
\midrule
\endhead
\(k\) & 3.841 & 6.635 & 10.828 \\
\bottomrule
\end{longtable}

\hypertarget{ux5df1ux77e5ux70b9uxaa21uxaux5728ux53ccux66f2ux7ebfuxacx2a2ux2212y2a2ux221211a1uxaux4e0aux76f4ux7ebfuxaluxaux4ea4uxacuxaux4e8euxapuxauxaquxaux4e24ux70b9ux76f4ux7ebfuxaapuxauxaaquxaux7684ux659cux7387ux4e4bux548cux4e3auxa0}{%
\paragraph{\texorpdfstring{己知点\(A(2,1)\)在双曲线\(C:{x^2 \over a^2} - {y^2 \over a^2 - 1}=1(a>1)\)上,直线\(l\)交\(C\)于\(P\),\(Q\)两点,直线\(AP\),\(AQ\)的斜率之和为\(0\)}{己知点A(2,1)在双曲线C:\{x\^{}2 \textbackslash over a\^{}2\} - \{y\^{}2 \textbackslash over a\^{}2 - 1\}=1(a\textgreater1)上,直线l交C于P,Q两点,直线AP,AQ的斜率之和为0}}\label{ux5df1ux77e5ux70b9uxaa21uxaux5728ux53ccux66f2ux7ebfuxacx2a2ux2212y2a2ux221211a1uxaux4e0aux76f4ux7ebfuxaluxaux4ea4uxacuxaux4e8euxapuxauxaquxaux4e24ux70b9ux76f4ux7ebfuxaapuxauxaaquxaux7684ux659cux7387ux4e4bux548cux4e3auxa0}}

\hypertarget{ux6c42uxaluxaux7684ux659cux7387}{%
\subparagraph{\texorpdfstring{求\(l\)的斜率;}{求l的斜率;}}\label{ux6c42uxaluxaux7684ux659cux7387}}

\hypertarget{ux82e5uxatanux2220paq22uxaux6c42uxaux25b3paquxaux7684ux9762ux79ef}{%
\subparagraph{\texorpdfstring{若\(\tan \ang PAQ = 2\sqrt2\),求\(\triangle PAQ\)的面积.}{若\textbackslash tan \textbackslash ang PAQ = 2\textbackslash sqrt2,求\textbackslash triangle PAQ的面积.}}\label{ux82e5uxatanux2220paq22uxaux6c42uxaux25b3paquxaux7684ux9762ux79ef}}

\hypertarget{ux5df2ux77e5ux51fdux6570uxafxexux2212axuxaux548cuxagxaxux2212lnxuxaux6709ux76f8ux540cux7684ux6700ux5c0fux503c}{%
\paragraph{\texorpdfstring{已知函数\(f(x)={\rm e}^x-ax\)和\(g(x)=ax-\ln x\)有相同的最小值}{已知函数f(x)=\{\textbackslash rm e\}\^{}x-ax和g(x)=ax-\textbackslash ln x有相同的最小值}}\label{ux5df2ux77e5ux51fdux6570uxafxexux2212axuxaux548cuxagxaxux2212lnxuxaux6709ux76f8ux540cux7684ux6700ux5c0fux503c}}

\hypertarget{ux6c42uxaauxa}{%
\subparagraph{\texorpdfstring{求\(a\);}{求a;}}\label{ux6c42uxaauxa}}

\hypertarget{ux8bc1ux660eux5b58ux5728ux76f4ux7ebfuxaybuxaux5176ux4e0eux4e24ux6761ux66f2ux7ebfuxayfxuxaux548cuxaygxuxaux5171ux6709ux4e09ux4e2aux4e0dux540cux7684ux4ea4ux70b9ux5e76ux4e14ux4eceux5de6ux5230ux53f3ux7684ux4e09ux4e2aux4ea4ux70b9ux7684ux6a2aux5750ux6807ux6210ux7b49ux5deeux6570ux5217}{%
\subparagraph{\texorpdfstring{证明:存在直线\(y=b\),其与两条曲线\(y=f(x)\)和\(y=g(x)\)共有三个不同的交点,并且从左到右的三个交点的横坐标成等差数列.}{证明:存在直线y=b,其与两条曲线y=f(x)和y=g(x)共有三个不同的交点,并且从左到右的三个交点的横坐标成等差数列.}}\label{ux8bc1ux660eux5b58ux5728ux76f4ux7ebfuxaybuxaux5176ux4e0eux4e24ux6761ux66f2ux7ebfuxayfxuxaux548cuxaygxuxaux5171ux6709ux4e09ux4e2aux4e0dux540cux7684ux4ea4ux70b9ux5e76ux4e14ux4eceux5de6ux5230ux53f3ux7684ux4e09ux4e2aux4ea4ux70b9ux7684ux6a2aux5750ux6807ux6210ux7b49ux5deeux6570ux5217}}

\end{document}
